\documentclass[10pt,a4paper,oneside,titlepage]{report}
\usepackage[utf8]{inputenc}
\usepackage[english,russian]{babel}
\usepackage{amsmath}
\usepackage{amsthm}
\usepackage{amssymb}
\usepackage{cmll}
\usepackage{enumerate}
\usepackage{stmaryrd}
\usepackage[left=2cm,right=2cm,top=2cm,bottom=2cm,bindingoffset=0cm]{geometry}
\usepackage{url}
\usepackage{multicol}
\usepackage{listingsutf8}
\usepackage{graphicx}
\graphicspath{{pictures/}}
\DeclareGraphicsExtensions{.pdf,.png,.jpg}

\lstset{%
	numbers = left
}

\title{Конспект по курсу Сети \thanks{Читаемый  в 2019-2020 годах}}
\author{Александра Лисицына \thanks{Студентка группы М3435}}

\theoremstyle{defenition}
\newtheorem*{defenition}{Определение}

\begin{document}

\maketitle

\tableofcontents

\clearpage	

\chapter{Рефакторинг}

\begin{defenition}
	Рефакторинг --- процесс изменения внутренней структуры программы, не затрагивающей ее внешнего поведения и имеющий целью облегчить понимание ее работы.
\end{defenition}

\begin{itemize}
	\item Цель рефакторинга --- заплатить технической долг.
	\item Результат рефакторинга --- чистый код и простой дизайн.
\end{itemize}

\section{Как стоит применять рефакторинг}

\begin{itemize}
	\item Серия небольших изменений
	\item Каждое исправляет какой-то один эффект
	\item Не добавляется никакой новой функциональности
	\item Все тесты проходят после каждого изменения
	\item Код становится проще и понятней после рефакторинга
\end{itemize}

\section{Когда стоит применять рефакторинг}

\begin{itemize}
	\item Правило трех ударов
	\begin{itemize}
		\item Первый раз вы просто пишете код
		\item Второй раз морщитесь от необходимости делать что-то похожее, но все пишете код
		\item На третий раз вы добавляете рефакторинг 
	\end{itemize}
    \item Когда добавляете новую функциональность
    \begin{itemize}
    	\item В процессе рефакторинга вы лучше понимаете существующий код и улучшаете его
    	\item После рефакторинга новый фичи добавлять проще
    \end{itemize}
\end{itemize}

\end{document}