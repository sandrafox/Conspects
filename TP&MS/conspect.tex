\documentclass[10pt,a4paper,oneside,titlepage]{book}
\usepackage[utf8]{inputenc}
\usepackage[english,russian]{babel}
\usepackage{amsmath}
\usepackage{amsthm}
\usepackage{amssymb}
\usepackage{cmll}
\usepackage{enumerate}
\usepackage{stmaryrd}
\usepackage[left=2cm,right=2cm,top=2cm,bottom=2cm,bindingoffset=0cm]{geometry}
\usepackage{url}
\usepackage{listingsutf8}
\usepackage{graphicx}
\graphicspath{{pictures/}}
\DeclareGraphicsExtensions{.pdf,.png,.jpg}

\title{Конспект по курсу: Теория вероятности и математическая статистика \thanks{читаемый Суслиной Ириной Александровной в 2018-2019 годах}}
\author{Александра Лисицына \thanks{студентка группы М3334}}

\theoremstyle{plain}
\newtheorem{theorem}{Теорема}
\newtheorem{lemma}{Лемма}
\newtheorem{predl}{Утверждение}

\theoremstyle{defenition}
\newtheorem{defenition}{Определение}[section]

\theoremstyle{remark}
\newtheorem*{remark}{Замечание}
\newtheorem*{example}{Пример}

\begin{document}

\maketitle

\tableofcontents

\clearpage

\chapter{Введение в теорию вероятностей}

\section{Введение}

\begin{defenition}
	{\bfseries Вероятностное пространство} --- стандартная модель теории вероятности $(\Omega, \Sigma, P)$.
\end{defenition}

\begin{defenition}
	{\bfseries Случайный эксперимент}(испытание) --- первичное понятие. Может быть повторён в идентичных условиях любое число раз. Результат эксперимента --- {\bfseries элементарный исход}.
	
	В результате эксперимента происходит один и только один исход. 
\end{defenition}

\begin{enumerate}
	\item [\textbf{Examples}]
	\item Случайное двукратное бросание монеты
	$\Omega=$\{ГГ, ГР, РГ, РР\} 
	\item Случайное извленчение двух карт из колоды в 36 карт. Порядок не важен
	$card(\Omega)=|\Omega|=C^2_{36}=18\times35$
	\item Бросание монеты до первого герба
	$\Omega=$ \{Г, РГ, РРГ, \ldots\}
	\item Поезда в метро ходят с интервалом 2 минуты. Человек появляется на платформе в произвольный момент. Сколько ему ожидать поезд?
	$\Omega=$ [0 мин, 2 мин)
\end{enumerate}

\section{Дискретная вероятностная схема}

Пусть $\Omega$ --- не более чем счётно

\begin{defenition}
	$\forall A\subseteq\Omega$ называют {\bfseries событием}. {\bfseries$\Sigma$} --- множество всех подможеств $\Omega$.
\end{defenition}

\begin{defenition}
	{\bfseries$P=P(A)$} --- счётно-аддитивная мера, если
	\begin{enumerate}
		\item $\forall\omega\in\Omega\quad P(\omega)\in[0, 1], p(\omega)\geqslant0$
		\item $\sum_{\omega\in\Omega}p(\omega)=1$
	\end{enumerate}
    $$
    P\colon\Omega\to\mathbb{R_+}
    $$
    $$
    \forall A\subseteq\Omega\quad P(A)=\sum_{\omega\in\Omega}p(\omega)
    $$
\end{defenition}

\begin{example}
	{\bfseries Классическая схема}
	$$
	|\Omega|=n
	$$
	исходы равновозможны: $$
	\omega\in\Omega p(\omega)=\frac{1}{\Omega}=\frac1n
	$$
	$$
	\sum_{\omega\in\Omega}p(\omega)=\frac1n\sum_{\omega\in\Omega}1=1
	$$
	В примере априорном 1
	
	$\Omega=$ \{ГГ, РГ, ГР, РР\}
	$$
	p(\omega_i)=\frac{1}{\Omega}=\frac14
	$$
	A --- присутствует герб
	$$
	p(A)=p(\mbox{ГГ})+p(\mbox{ГР})+p(\mbox{РГ})=\frac34
	$$
\end{example}

\section{Несчётное вероятностное пространство}

\begin{defenition}
Пусть $\Omega$ --- более чем счётно $\Rightarrow$ {\bfseries$\Sigma$} --- выделенная $\sigma$-алгебра из пдмножеств $\Omega$. 

$A\in\Sigma\quad\Rightarrow A$ называют {\bfseries соытием}.
\end{defenition}

Для определения $\mathcal{A}$ --- алгебра, $\Sigma$ --- $\sigma$-алгебра нужны действия:
\begin{enumerate}
	\item Сложение ($A+B$)
	\item Умножение ($A\cdot B$)
	\item Разность ($A\backslash B$)
\end{enumerate}

\subsection{Свойства действий}

\begin{enumerate}
	\item $A+B=B+A$, $(A+B)+C=A+(B+C)$
	\item $A\cdot B=B\cdot A$, $(A\cdot B)\cdot C=A\cdot(B\cdot C)$
	\item $A\cdot\Omega=A$, $A\cdot\emptyset=\emptyset$, $A+\Omega=\Omega$, $A+\emptyset=A$
	\item $\overline{A}=\Omega\backslash A$, $A\cdot\overline{A}=\emptyset$
	\item $A+\overline{A}=\Omega$
	\item $A\cdot B\subseteq A$, $A\cdot B\subseteq B$
	\item $\overline{\overline{A}}=A$
	\item $A\backslash B=A\cdot\overline{B}$
	\item $\overline{A+B}=\overline{A}\cdot\overline{B}$
	\item $\overline{A\cdot B}=\overline{A}+\overline{B}$
	\item $(A+B)\cdot C=A\cdot C + B\cdot C$
	\item $(A\backslash B)\cdot C=AC\backslash BC$
\end{enumerate}

\begin{remark}
	Свойства 9--11 обобщаются на любое число множеств
	\begin{enumerate}\addtocounter{enumi}{8}
		\item $\overline{\sum\limits^{\infty}_{i=1}A_i}=\prod\limits_{i=1}^{\infty}\overline{A_i}$
		\item $\overline{\prod\limits_{i=1}^{\infty}A_i}=\sum\limits_{i=1}^{\infty}\overline{A_i}$
		\item $B\sum\limits_{i=1}^{\infty}A_i=\sum\limits_{i=1}^{\infty}BA_i$
	\end{enumerate}
\end{remark}

\begin{defenition}
	$\mathcal{A}$ называют алгеброй множеств если
	\begin{enumerate}
		\item $\Omega$ --- объект $\mathcal{A}$
		\item Пусть $A\in\mathcal{A}\Rightarrow\overline{A}\in\mathcal{A}$
		\item Пусть $A, B\in\mathcal{A}\Rightarrow A+B\in\mathcal{A}$
	\end{enumerate}
\end{defenition}

\begin{predl}
	Алгебра замкнута относительно действий в конечном числе
\end{predl}
\begin{proof}[Доказательство утверждения]
	\begin{enumerate}
		\item $\forall A,B\in\mathcal{A}\quad A\cdot B=\overline{\overline{A}+\overline{B}}\Rightarrow A\cdot B\in\mathcal{A}$
		\item $\forall A,B\in\mathcal{A}\quad A\backslash B=A\cdot\overline{B}$
	\end{enumerate}
\end{proof}

\begin{defenition}
	$\Sigma$ называют {\bfseries$\sigma$-алгеброй множеств} $\Omega$, если $\Sigma$ --- алгебра и $A_i\in\Sigma$, $i=1, 2, \ldots \Rightarrow\sum\limits_iA_i\in\Sigma$
\end{defenition}

\begin{defenition}
	<$\Omega$, $\Sigma$> называют {\bfseries измеримым пространством}.
\end{defenition}

\subsection{Связь $\mathcal{A}$ и $\Sigma$}

\begin{enumerate}
	\item Пусть $|\Omega|=n$
	
	$\mathcal{A}$ --- содержит все элементы $\Omega \Rightarrow \mathcal{A}=\Sigma$
	\item $|\Omega|=\mathbb{N}$
	
	Пусть $\mathcal{A}$ --- содежит все элемнты $\Omega$
	
	$\mathcal{A}$ не содержит, например, всех чётных натуральных чисел
	\item $\Omega$ более чем счётно
	
	Пусть $\Omega=\mathbb{R}\Rightarrow\Sigma$ порождается множеством интервалов (абсолютн любых). Если только [a, b), то она называется {\bfseries борелевской} $\sigma$-алгеброй.
	\begin{remark}
		Борелевская $\sigma$-алгебра отличается от Лебега тем, что она не замкнута.
	\end{remark}
    Пусть $\Omega=\mathbb{R^n}\Rightarrow\Sigma$ порождается n-мерными кубами: $[a_1, b_1)\times[a_2,b_2)\times\ldots$
    
    Борелевская алгебра для $\mathbb{R}$ представляет собой множество элементов вида: $A=\sum\limits_{i=1}^n[a_i,b_i)\quad[a_i,b_i]\bigcap\limits^{i\ne j}[a_j,b_j)=\emptyset$
\end{enumerate}

\begin{defenition}
	\begin{enumerate}
		\item $\Omega=D$ --- {\bfseries достоверное} событие.
		\item $\emptyset$ --- {\bfseries невозможное} событие.
		\item $A\cdot B=\emptyset\Rightarrow A$ и $B$ называют {\bfseries несовместными}.
		\item $\overline{A}$ --- {\bfseries противоположное} событие.
	\end{enumerate}
\end{defenition}
\end{document}
