\documentclass[10pt,a4paper,oneside]{article}
\usepackage[utf8]{inputenc}
\usepackage[english,russian]{babel}
\usepackage{amsmath}
\usepackage{amsthm}
\usepackage{amssymb}
\usepackage{cmll}
\usepackage{enumerate}
\usepackage{stmaryrd}
\usepackage[left=2cm,right=2cm,top=2cm,bottom=2cm,bindingoffset=0cm]{geometry}
\usepackage{url}

\theoremstyle{plain}
\newtheorem*{theorem}{Теорема}
\newtheorem{lemma}{Лемма}[section]

\theoremstyle{defenition}
\newtheorem*{defenition}{Определение}

\begin{document}

\section{Бестиповое лямбда-исчисление. Общие определения (альфа-эквивалентность, бета-редукция, бета-эквивалентность). Параллельная бета-редукция. Теорема Чёрча-Россера.}
\begin{defenition}
	$A=_{\alpha}B$
	\begin{enumerate}
		\item $A\equiv x \wedge B\equiv y \wedge x\equiv y$
		\item $A\equiv P_1 Q_1 \wedge B\equiv P_2 Q_2 \wedge P_1=_{\alpha}P_2 \wedge Q_1=_{\alpha}Q_2$
		\item $A\equiv\lambda x.P \wedge B\equiv\lambda y.Q \wedge P[x\colon=t]=_{\alpha}Q[y\colon=t]$
	\end{enumerate}
\end{defenition}

\begin{defenition}
	$A\to_{\beta}B$
	\begin{enumerate}
		\item $A\equiv P_1 Q_1 \wedge B\equiv P_2 Q_2$
		
		      $P_1=_{\alpha}P_2 \wedge Q_1\to_{\beta}Q_2$
		      или
		      $P_1\to_{\beta}P_2 \wedge Q_1=_{\alpha}Q_2$
		\item $A\equiv(\lambda x.P) Q \wedge B\equiv P[x\colon=Q]$
	\end{enumerate}
\end{defenition}

\begin{defenition}
	$A=_{\beta}B$ --- рефлексивное, транзитивное, симметричное замыкание $\to_{\beta}$.
\end{defenition}

\begin{defenition}
	$A\rightrightarrows_{\beta}B$
	\begin{enumerate}
		\item $A=_{\alpha}B$
		\item $A\equiv P_1 Q_1 \wedge B\equiv P_2 Q_2 \wedge P_1\rightrightarrows_{\beta}P_2 \wedge Q_1\rightrightarrows_{\beta}Q_2$
		\item $A=_{\alpha}(\lambda x.P) Q \wedge B=_{\alpha}P[x\colon=Q]$
	\end{enumerate}
\end{defenition}

\begin{theorem}[Чёрча--Россера]
	$\twoheadrightarrow_{\beta}$ обладает ромбовидным свойством.
\end{theorem}

\section{Булевские значения, чёрчевские нумералы, упорядоченные пары}

$T\equiv\lambda a.\lambda b.a$

$F\equiv\lambda a.\lambda b.b$

$If\equiv\lambda c.\lambda t.\lambda e.(c t) e$

$\overline{n}\equiv\lambda f.\lambda x.f^{(n)} x$

$(+1)\equiv\lambda n.\lambda f.\lambda x.n f (f x)$

$makePair\equiv\lambda a.\lambda b.\lambda s.s a b$

$First\equiv\lambda p.p T$

$Second\equiv\lambda p.p F$
\end{document}